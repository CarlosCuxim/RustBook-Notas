\documentclass[pagecolor=true, pagesize=a5paper, stretchmode]{qx-files/qx-notes}

\usepackage{qx-files/qx-listings}
\setmainlanguage{rust}



% IDIOMA ========================================================
\usepackage[spanish, mexico]{babel}



% DATOS =========================================================
\title{Test file}
\author{Qx}
\date{\today}



\usepackage{lipsum}


\begin{document}
  \maketitle

  Esta madre es un ejemplo de un listing del lenguaje rust. Estoy tratando de ver
  como se comporta con respecto al texto y los saltos de lineas

  \begin{codeblock}
    // Code taken from the Rust Book. Guessing game of section 2

    use rand::Rng;
    use std::cmp::Ordering;
    use std::io;

    fn main() {
      println!("Guess the number!");

      let secret_number = rand::thread_rng().gen_range(1..=100);

      loop {
        println!("Please input your guess.");

        let mut guess = String::new();

        io::stdin()
            .read_line(&mut guess)
            .expect("Failed to read line");

        let guess: u32 = match guess.trim().parse() {
          Ok(num) => num,
          Err(_) => continue,
        };

        println!("You guessed: {guess}");

        match guess.cmp(&secret_number) {
          Ordering::Less => println!("Too small!"),
          Ordering::Greater => println!("Too big!"),
          Ordering::Equal => {
            println!("You win!");
            break;
          }
        }
      }
    }
  \end{codeblock}

  \begin{codeblock}[c++]
    // Necessary header files for input output functions
    #include <iostream>
    using namespace std;

    // main() function: where the execution of
    // C++ program begins
    int main() {
      
        // This statement prints "Hello World"
        cout << "Hello World";

        return 0;
    }
  \end{codeblock}


  \begin{verbblock}
    0 _______ 10 ______ 20 ______ 30 ______ 40 ______ 50 ______ 60 ______ 70 _______
  \end{verbblock}

  Supongamos que tenemos un parrafo, me gustaría observar como es que
  \codeline{let mut s = String::new("Hello");} se vería a la mitad de un parrafo, no estoy segura de que es lo que busco, pero si de lo que no se ve.
  
  Tengo que probar el verb \verbline{verb&text} para ver si funciona, tambien tengo que 
  escribir \codeline[python]{print("Hello")}.

  Ahora probaré agregar un listing mediante un comando

  \inputcode{listings/hello_world.rs}
  \inputcode[python]{listings/py_hello.py}
  \inputverb{listings/output.txt}


\end{document}

