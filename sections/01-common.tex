\chapter{Conceptos Comunes de Programación}



\section{Estructura básica} % --------------------

\paragraph{Tipo de lenguaje.} Rust es un lenguaje compilado de tipado fuerte. Es decir,
cada variable y función tiene un tipo asociado e inmutable además de que para ejecutar
un programa de rust se necesita crear un binario mediante el programa \codeline{rustc}.

\paragraph{Cargo.} Es un programa que sirve para crear y ejecutar proyectos de rust. Los
comandos básicos son: \codeline{cargo new} para crear un nuevo proyecto;
\codeline{cargo run} para correr el programa; \codeline{cargo build} para crear un
binario del proyecto.

\paragraph{Hola mundo.} La estructura básica de un ``hola mundo'' en Rust es similar a C,
basta con una función \codeline{main()} que contiene las instrucciones que se ejecutarán.

Todas las instrucciones deben terminar ya sea con \codeline{;} o \codeline|}|, en el caso
de que se esté terminando un bloque. El indentado y espaciado no son importantes para la
ejecución del código.

  \inputexample{hello_world}



\section{Variables y mutabilidad} % --------------

\paragraph{Variables} Se definen usando la palabra clave \codeline{let} y opcionalmente
asignando un valor usando \codeline{=}.

  \inputexample{variables_001}

\paragraph{Variables inmutables} Estas variables no permiten modificar su valor una vez
asignadas. Si se decide asignar otro valor con \codeline{=} se mostrará un error. 

  \inputexample{variables_002}